\chapter{Pesquisa-ação}
\label{pesquisa_acao}
	\section{Descrição da organização}

		A organização a ser tratada neste trabalho refere-se aos grupos de trabalho das disciplinas de Gestão de Projetos 
		e Portfólio (GPP) e Metodologias de Desenvolvimento de Software (MDS), do curso de Engenharia de \textit{Software} 
		da Universidade de Brasília.

		A organização trabalha na produção de software com vários projetos, os quais possuem as seguintes características:

		\begin{itemize}
			\item Uso de dados abertos;
			\item Aplicações WEB e aplicativos \textit{mobile};
			\item Equipe composta de dois times:
				\subitem 1 time de desenvolvimento;
				\subitem 1 equipe de gerência;
			\item Equipes pequenas e inexperientes;
			\item Prazo fixo;
		\end{itemize}

		\subsection{Objeto da ação}

		Para a realização da pesquisa-ação será considerado um projeto em andamento da organização que possui as seguintes características:


		\begin{itemize}
			\item Aplicativo \textit{mobile};
			\item Equipe composta de dois times:
				\subitem 1 time de desenvolvimento com 6 pessoas;
				\subitem 1 equipe de gerência com 6 pessoas;
			\item Membros inexperientes na tecnologia.
			\item Uso de metodologia ágil;
			\item Quantidade de \textit{Sprints}: 7
		\end{itemize}



\section{Diagnóstico}
	



	\subsection{Realização do diagnóstico}

	%Entrevistas


\section{Planejamento da ação}
\section{Execução da ação}
\section{Avaliação da ação}