\chapter{Pesquisa-ação}
\label{pesquisa_acao}
	\section{Descrição da organização}

		A organização a ser tratada neste trabalho refere-se aos grupos de trabalho das disciplinas de Gestão de Projetos
		e Portfólio (GPP) e Metodologias de Desenvolvimento de Software (MDS), do curso de Engenharia de \textit{Software}
		da Universidade de Brasília.

		A organização trabalha na produção de software com vários projetos, os quais possuem as seguintes características:

		\begin{itemize}
			\item Uso de dados abertos;
			\item Aplicações WEB e aplicativos \textit{mobile};
			\item Equipe composta de dois times:
				\subitem 1 time de desenvolvimento;
				\subitem 1 equipe de gerência;
			\item Equipes pequenas e inexperientes;
			\item Prazo fixo;
		\end{itemize}

		\subsection{Objeto da ação}

		Para a realização da pesquisa-ação será considerado um projeto em andamento da organização que possui as seguintes características:


		\begin{itemize}
			\item Aplicativo \textit{mobile};
			\item Equipe composta de dois times:
				\subitem 1 time de desenvolvimento com 6 pessoas;
				\subitem 1 equipe de gerência com 6 pessoas;
			\item Membros inexperientes na tecnologia.
			\item Uso de metodologia ágil;
			\item Quantidade de \textit{Sprints}: 7
		\end{itemize}



\section{Diagnóstico}




	\subsection{Realização do diagnóstico}

	%Entrevistas


	\section{Planejamento da ação}

		SOMMERVILLE (2007) destaca que a medição de software preocupa-se com a derivação de um valor numérico ou o perfil para um atributo de um componente de software, sistema ou processo. Comparando esses valores entre si e com os padrões que se aplicam a toda a organização, pode-se tirar conclusões sobre a qualidade do software ou avaliar a eficácia dos métodos, das ferramentas e dos processos de software. Como dizia Tom de Marco (1982), "Não se pode gerenciar o que não se pode medir".

		Softwares podem ser medidos (ou estimados) de diversas formas, como tamanho, custo e esforço. E, para a medição de um produto, podem ser colhidas diferentes métricas. Por exemplo, para a medição de tamanho na etapa de levantamento de requisitos, pode-se utilizar como métrica o número de requisitos especificados. Já na fase de projeto, o tamanho pode ser medido em função do número de classes e, na fase de codificação, a partir no número de linhas de código fonte.

		http://www.devmedia.com.br/artigo-engenharia-de-software-21-metricas-de-software/15776

		https://www.assembla.com/spaces/procsw/documents/d92WTA434r36xHeJe5cbLA/download/d92WTA434r36xHeJe5cbLA

		BASSI (2012) cita a metodologia Lean como sendo distribuída nos princípios de eliminar o desperdício, amplificar o aprendizado, adiar comprometimentos e manter a flexibilidade, entregar rápido, tornar a equipe responsável, construir integridade e visualizar o todo. O princípio de eliminar o desperdício foca no sentido de que o desperdício em si pode acontecer em vários sentidos, entre eles: dinheiro, recursos, tempo, esforço e espaço. Cada etapa e atividade realizada no processo devem contribuir para que seja possível construir um produto final com menos custo, mais rapidez e com qualidade.

		Para esse trabalho foram escolhidas três métricas principais: esforço, tamanho e tempo. A métrica tamanho, como já foi citado, pode ser alcançada de diversas formas. Para o contexto do trabalho analisaremos o tamanho realizando a contagem de número de linhas de código que, por sua vez, auxiliará a estimar o esforço a ser considerado para a obtenção de um produto a ser desenvolvido. O tempo estudado será aquele estimado a se realizar determinada tarefa.

\section{Execução da ação}
\section{Avaliação da ação}