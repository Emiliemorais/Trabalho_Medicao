\subsection{Ciclo 1}

	Para o ciclo 1 foi definido que a ação seria de planejamento não havendo a avaliação da ação para este ciclo. Assim, as atividades a serem
	executadas estão descritas na subseção \ref{sub:planejamento} (Planejamento da ação) e sua execução na subseção \ref{sub:execucao} (Execução da ação).

	\subsubsection{Planejamento da ação}
		\label{sub:planejamento}

		No ciclo 1 as atividades para serem executadas são:

		\begin{enumerate}

			\item Elaborar estratégia de melhoria para a equipe;

			\item Elaborar proposta de avaliação da estratégia;

			\item Elaborar plano de medição;

			\item Coletar as métricas definidas da situação atual;


		\end{enumerate}


	\subsubsection{Execução da ação}
		\label{sub:execucao}

		\subsubsubsection{Estratégia de melhoria}

			Considerando os aspectos identificados como maiores influenciadores dos atrasos no projeto, a proposta de uma maneira geral deve buscar:

			\begin{itemize}

				\item Mitigar os erros nas estimativas;
				\item Diminuir a dificuldade da equipe na linguagem;
				\item Motivar a equipe a trabalhar cedo;
				\item Melhorar a comunicação da equipe;
			
			\end{itemize}

			\textbf{Mitigar os erros nas estimativas}

				Para mitigar os erros nas estimativas, a proposta é reestimar os pontos atribuídos às histórias no começo da \textit{Sprint} quando ela terminar. Possibilitando ao time perceber se as estimativas realizadas refletiram a realidade. A partir dessa percepção poder melhorar as estimativas da próxima \textit{Sprint}.

			\textbf{Diminuir a dificuldade da equipe na linguagem}

				A partir da avaliação realizada com as métricas definidas é proposta a realização de treinamentos nas áreas de maior dificuldade.

			\textbf{Motivar a equipe a trabalhar cedo}

				A partir da análise das métricas apresentar os resultados ao time de forma a motivá-los a começarem a trabalhar mais cedo para evitar os atrasos.

			\textbf{Melhorar a comunicação da equipe}

				Propor dinâmicas de integração da equipe, reforçar as reuniões diárias provenientes da metodologia.


		\subsubsubsection{Proposta de avaliação da estratégia}

				Para avaliar a efetividade da estratégia será observado se os atrasos continuaram e serão coletadas as métricas contidas no plano
				de medição.


		\subsubsubsection{Plano de medição}
		
		 Para a elaboração do processo de medição foram consideradas atividades propostas tanto pela ISO/IEC 15939:2007 
		 quanto pelo método GQM.

		\subsubsubsection{Métricas coletadas da situação atual}
		
		 



\subsection{Ciclo 2..n}

	O tempo para a realização do trabalho não permitiu a execução de mais de um ciclo. Todavia,
	sabe-se que os próximos ciclos seriam a execução da estratégia proposta no ciclo 1 e a cada avaliação realizada
	seriam endereçadas melhorias a fim de refinar a estratégia.

	A quantidade de ciclos foi estabelecida como indefinida, pois seriam executados ciclos até que a estratégia 
	se tornasse estável.

