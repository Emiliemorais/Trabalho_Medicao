\subsection{Ciclo 1}

	Para o ciclo 1 foi definido que a ação seria de planejamento não havendo a avaliação da ação para este ciclo. Assim, as atividades a serem
	executadas estão descritas na subseção \ref{sub:planejamento} (Planejamento da ação) e sua execução na subseção \ref{sub:execucao} (Execução da ação).

	\subsubsection{Planejamento da ação}
		\label{sub:planejamento}

		No ciclo 1 as atividades para serem executadas são:

		\begin{enumerate}

			\item Elaborar estratégia de melhoria para a equipe;

			\item Elaborar proposta de avaliação da estratégia;

			\item Elaborar plano de medição;


		\end{enumerate}


	\subsubsection{Execução da ação}
		\label{sub:execucao}

		\subsubsubsection{Estratégia de melhoria}

		\subsubsubsection{Proposta de avaliação da estratégia}

		\subsubsubsection{Plano de medição}

\subsection{Ciclo 2..n}

	O tempo para a realização do trabalho não permitiu a execução de mais de um ciclo. Todavia,
	sabe-se que os próximos ciclos seriam a execução da estratégia proposta no ciclo 1 e a cada avaliação realizada
	seriam endereçadas melhorias a fim de refinar a estratégia.

	A quantidade de ciclos foi estabelecida como indefinida, pois seriam executados ciclos até que a estratégia 
	se tornasse estável.

