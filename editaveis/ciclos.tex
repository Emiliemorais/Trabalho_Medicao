\subsection{Ciclo 1}

	Para o ciclo 1 foi definido que a ação seria de planejamento não havendo a avaliação da ação para este ciclo. Assim, as atividades a serem
	executadas estão descritas na subseção \ref{sub:planejamento} (Planejamento da ação) e sua execução na subseção \ref{sub:execucao} (Execução da ação).

	\subsubsection{Planejamento da ação}
		\label{sub:planejamento}

		No ciclo 1 as atividades para serem executadas são:

		\begin{enumerate}

			\item Elaborar estratégia de melhoria para a equipe;

			\item Elaborar plano de medição;

			\item Elaborar proposta de avaliação da estratégia;

			\item Coletar as métricas definidas da situação atual.

		\end{enumerate}


	\subsubsection{Execução da ação}
	\label{sub:execucao}
  
	    Nesta seção são apresentados os resultados das atividades planejadas da ação.
	
	    \subsubsubsection{Estratégia de melhoria}

		Considerando os aspectos identificados como maiores influenciadores dos atrasos no projeto,
		a proposta de uma maneira geral deve buscar:

		\begin{itemize}

		  \item \textbf{Mitigar os erros nas estimativas};
		    
		    \subitem Para mitigar os erros nas estimativas, a proposta é reestimar os pontos atribuídos às histórias
			      no começo da \textit{Sprint} quando ela terminar. Possibilitando ao time perceber
			      se as estimativas realizadas refletiram a realidade. A partir dessa percepção poder
			      melhorar as estimativas da próxima \textit{Sprint}.
		    
		  \item \textbf{Diminuir a dificuldade da equipe na linguagem};
		    
		    \subitem A partir da avaliação realizada com as métricas definidas é proposta a realização de
			      treinamentos nas áreas de maior dificuldade.
		    
		  \item \textbf{Motivar a equipe a trabalhar cedo};
		  
		    \subitem A partir da análise das métricas apresentar os resultados ao time de forma a
			      motivá-los a começarem a trabalhar mais cedo para evitar os atrasos.
		    
		  \item \textbf{Melhorar a comunicação da equipe}.
		    
		    \subitem Propor dinâmicas de integração da equipe, reforçar as reuniões diárias
			      provenientes da metodologia.
	      
		\end{itemize}
		
		Todas as estratégias apresentadas devem ser aplicadas de acordo com o resultado da análise das métricas.

	    \subsubsubsection{Proposta de avaliação da estratégia}

		Para avaliar a efetividade da estratégia, será observado se os atrasos continuaram e 
		serão comparados os resultados das métricas obtidas em cada ciclo, para avaliar se os valores
		das métricas melhoraram ou não.

	    \subsubsubsection{Plano de medição}
	    
		Para a elaboração do processo de medição que dará suporte à estratégia, foram consideradas atividades
		propostas tanto pela ISO/IEC 15939:2007 quanto pelo método GQM.
		O plano de medição proposto pode ser visto no Apêndice \ref{plano_medicao}.

	    \subsubsubsection{Métricas coletadas da situação atual}
	      
		As métricas propostas no plano de medição foram coletadas para a \textit{Sprint} 0 do
		projeto proposto, a título de exemplificar o preenchimento do relatório modelo para a análise
		dos dados coletados, com dados reais do projeto. O resultado desta coleta pode ser
		visto no Apêndice \ref{relatorio_1}.

\subsection{Ciclo 2..n}

	O tempo para a realização do trabalho não permitiu a execução de mais de um ciclo. Todavia,
	sabe-se que os próximos ciclos seriam a execução da estratégia proposta no ciclo 1 e a cada avaliação realizada
	seriam endereçadas melhorias a fim de refinar a estratégia.

	A quantidade de ciclos foi estabelecida como indefinida, pois seriam executados ciclos até que a estratégia 
	se tornasse estável.

