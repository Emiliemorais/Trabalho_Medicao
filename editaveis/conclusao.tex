\chapter{Conclusão}

Este trabalho teve o intuito de, por meio do uso de uma pesquisa-ação e revisão bibliográfica, analisar entregas de incrementos de \textit{software} por \textit{sprints} afim de minimizar os atrasos nessas entregas sob o ponto de vista de gerente de um projeto de GPP/MDS proposto.

Através da aplicação de um questionário no grupo de GPP/MDS foi possível a percepção de alguns fatores principais os quais exercem uma influência considerável nos atrasos nas entregas por \textit{sprints}. Foram levantados quatro fatores principais:

\begin{itemize}
  \item Erros nas estimativas;
  \item Dificuldade na linguagem;
  \item Procrastinação;
  \item Dificuldade de comunicação.
\end{itemize}

Ao início da pesquisa-ação foi elaborado o diagnóstico da equipe afim de se conhecer a situação atual da do trabalho realizado, e aspectos como o \textit{velocity}. Para compor os ciclos foram definidas três atividades principais, o planejamento, a execução e a avaliação da ação.

No primeiro ciclo foram executadas as atividades de planejamento e execução da ação, não sendo possível executar a avaliação do primeiro ciclo e os ciclos seguintes - quantidade indefinida, visto que seriam executados até que a estratégia se tornasse estável - dado o tempo fechado da disciplina.

No plano de medição elaborado foi definido o time de medição, escopo de medição, programa de medição e os procedimentos a serem adotados, sendo estes a coleta de dados, análise dos dados e a comunicação dos dados.

Para auxiliar a execução e a avaliação do ciclo, elaborou-se um \textit{template} de relatório (ref apendice c)  visando a coleta, armazenamento e análise das métricas utilizadas para responder as questões presentes no plano de medição, já sendo utilizado no ciclo 1. (ref apendice d)

Após a realização do trabalho, se tornou evidente a necessidade da elaboração de um plano de medição em um projeto. Mesmo não tendo as limitações orçamentárias presentes em uma organização, a disciplina tornou possível a análise de algumas dificuldades encontradas, tais como a identificação das questões e seu refinamento.

Além disso, como dois dos integrantes da equipe de medição também eram integrantes do objeto de estudo (grupo de GPP/MDS), a experiência com a pesquisa-ação foi satisfatória, de modo que o \textit{feedback} para a equipe era instantâneo e foi possível ter um bom aproveitamento da metodologia. Dessa forma, houve a pesquisa - dentro do âmbito da disciplina de medição - e também foram realizadas melhorias dentro do objeto de estudo.
