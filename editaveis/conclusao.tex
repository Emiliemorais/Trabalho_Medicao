\chapter{Conclusão}

Este trabalho teve o intuito de analisar entregas de incrementos de \textit{software} por \textit{sprints},
por meio do uso de uma pesquisa-ação e revisão bibliográfica, 
afim de minimizar os atrasos nessas entregas sob o ponto de vista de gerente do projeto proposto.
O objetivo geral proposto para o trabalho não foi atingido por falta de tempo para executar os demais ciclos
da pesquisa-ação, podendo ser alvo de um trabalho futuro.

Através da aplicação de um questionário na equipe do projeto proposto foi possível a percepção de alguns fatores principais
os quais exercem uma influência considerável nos atrasos nas entregas por \textit{sprints}. Foram identificados quatro
fatores principais para o projeto em questão:

\begin{itemize}
  \item Procrastinação;
  \item Erros nas estimativas;
  \item Dificuldade na linguagem;
  \item Dificuldade de comunicação.
\end{itemize}

% Ao início da pesquisa-ação foi elaborado o diagnóstico da equipe afim de se conhecer a situação atual da do
% trabalho realizado, e aspectos como o \textit{velocity}. Para compor os ciclos foram definidas três atividades
% principais, o planejamento, a execução e a avaliação da ação.

No primeiro ciclo da pesquisa-ação definida foram executadas as atividades de planejamento e execução da ação, não sendo possível
executar a avaliação do primeiro ciclo e os ciclos seguintes - quantidade indefinida, visto que seriam
executados até que a estratégia se tornasse estável - dado o tempo fechado da disciplina.

A partir dos fatores identificados, foi elaborada uma estratégia de melhoria que atacasse cada fator,
bem como também foi elaborada uma proposta de avaliação da estratégia definida.
Para fornecer o suporte à estratégia de melhoria proposta, foi elaborado um plano de medição contendo
o programa de medição e os procedimentos a serem adotados para a coleta e análise dos dados.
A avaliação da estratégia definida não foi realizada por ter sido executado apenas o primeiro ciclo da pesquisa-ação,
que consistia na elaboração da estratégia e da proposta de avaliação desta estratégia.

Após a realização do trabalho, se tornou evidente a necessidade da elaboração de um plano de medição em
um projeto, para permitir o acompanhamento quantitativo (e qualitativo) do projeto e fornecer uma base
para propor e avaliar melhorias.
% Mesmo não tendo as limitações orçamentárias presentes em uma organização, a disciplina tornou
% possível a análise de algumas dificuldades encontradas, tais como a identificação das questões e seu refinamento.

Além disso, como dois dos integrantes da equipe de medição também eram integrantes do objeto de estudo
(grupo de GPP/MDS), a experiência com a pesquisa-ação foi satisfatória, de modo que o \textit{feedback}
para a equipe era instantâneo e foi possível ter um bom aproveitamento da metodologia. Dessa forma,
houve a pesquisa - dentro do âmbito da disciplina de medição - e também foram realizadas melhorias dentro do objeto de estudo.
