\begin{apendicesenv}

\partapendices

  	\chapter{Questionário}
  		\label{apendice:questionario}
  		\begin{figure}[!htb]
  		\center
  		\includegraphics[scale=0.7]{figuras/questionario.pdf}
  		\end{figure}
 
  		\begin{figure}[!htb]
  		\center
  		\includegraphics[scale=0.7]{figuras/questionario2.pdf}
  		\end{figure}
 
  		\vfill
  		\pagebreak
 
  		\section{Respostas do Questionário}
  			\label{apendice:graficos}
  			\begin{figure}[!htb]
  			\center
  			\includegraphics[scale=0.9]{figuras/grafico_1.png}
  			\end{figure}
 
  			\begin{figure}[!htb]
  			\center
  			\includegraphics[scale=0.9]{figuras/grafico_2.png}
  			\end{figure}
 
  			\begin{figure}[!htb]
  			\center
  			\includegraphics[scale=0.9]{figuras/grafico_3.png}
  			\end{figure}
 
  			\begin{figure}[!htb]
  			\center
  			\includegraphics[scale=0.9]{figuras/grafico_4.png}
  			\end{figure}
 
  			\begin{figure}[!htb]
  			\center
  			\includegraphics[scale=0.9]{figuras/grafico_5.png}
  			\end{figure}
 
  			\begin{figure}[!htb]
  			\center
  			\includegraphics[scale=0.9]{figuras/grafico_6.png}
  			\end{figure}
 
  			\begin{figure}[!htb]
  			\center
  			\includegraphics[scale=0.9]{figuras/grafico_7.png}
  			\end{figure}
 
  			\begin{figure}[!htb]
  			\center
  			\includegraphics[scale=0.9]{figuras/grafico_8.png}
  			\end{figure}

	\chapter{Plano de medição}

	  \section{Time de medição (Time GQM)}
    
	O time que realizará as medições, bem como executar o presente plano, é composto pelos seguintes integrantes,
	sendo os dois primeiros integrantes componentes da equipe de gerência do projeto alvo:
	
	\begin{itemize}
	 \item Emilie Morais;
	 \item Ítalo Paiva;
	 \item Leonardo Cambraia;
	 \item Lucas Costa;
	 \item Omar Faria.
	\end{itemize}

	O comprometimento com o processo de medição é facilmente estabelecido, uma vez que dois integrantes do time de GQM 
	fazem parte da gerência do projeto em questão.
      
    \section{Escopo da medição e caracterização da organização}
    
      A organização e o projeto que serão tratados neste plano estão descritos nas seções 
      \ref{organizacao} e \ref{diagnostico}, respectivamente.
    
    \section{Programa de medição}
      
      \subsection{Objetivos de medição}
      
      \subsection{Questões}
      
      \subsection{Métricas}
      
    \section{Procedimentos}
    
      \subsection{Coleta de dados}
      
      \subsection{Análise dos dados}
      
      \subsection{Comunicação dos resultados}
	      
	      
	\chapter{Relatório da análise das métricas}
	\label{relatorio_template}

	Este relatório tem como objetivo servir como apoio para a coleta, armazenamento e análise das métricas que serão 
	utilizadas para responder as questões levantadas.
	\\
	

	M4 (Média de pontos alocados por \textit{sprint}): \underline{ }\underline{ }\underline{ }
	
	M5 (Pontuação das histórias):  \underline{ }\underline{ }\underline{ }
	
	M6 (Repontuação das histórias): \underline{ }\underline{ }\underline{ }
	
	M7 (Erro na estimativa por história): \underline{ }\underline{ }\underline{ }
	
	M8 (Precisão da estimativa): \underline{ }\underline{ }\underline{ }
	
	M9 (Erro médio nas estimativas): \underline{ }\underline{ }\underline{ }
	\\
	\\
	\textbf{\textit{Sprint}: \underline{ }\underline{ }\underline{ }}
	
	\begin{itemize}
	  
	  \item Questão 1 - Qual a influência da procrastinação no atraso das entregas?
	    \subitem Valor coletado - M1 (\textit{Velocity} diário):
	      \subsubitem Sexta -
	      \subsubitem Sábado -
	      \subsubitem Domingo -
	      \subsubitem Segunda -
	      \subsubitem Terça -
	      \subsubitem Quarta -
	      \subsubitem Quinta -
	      \subsubitem Sexta - 
	    \subitem Valor coletado - M2 (Média de pontos realizados por dia): 
	    \subitem Valor coletado - M3 (Pontos planejados por \textit{sprint}):
	    \subitem \textit{Análise}:
	    \\
	    \\
	    \vfill
	    \pagebreak

	  \item Questão 2 - Qual a influência dos erros das estimativas no atraso das entregas? 
	    \subitem Valor coletado - M3 (Pontos planejados por \textit{sprint}):
	    \subitem Valor coletado - M10 (\textit{Velocity} da equipe por \textit{sprint}):
	    \subitem \textit{Análise}:
	    \\
	    \\
	    
	  \item Questão 3 - Qual a influência da dificuldade na linguagem no atraso das entregas?
	    \subitem \textit{Análise}:
	  \\
	  \\
	  
	  \item Questão 4 - Qual a influência da dificuldade de comunicação no atraso das entregas?
	    \subitem \textit{Análise}:
	  \\
	  \\
	  
	\end{itemize}
	
  \chapter{Relatório da análise das métricas - Ciclo 1}
  \label{relatorio_1}
      
	Este relatório apresenta dados parciais coletados na Sprint 0, seguindo o plano de medição proposto.
	
	\section{Métricas gerais considerando a Sprint 0}
	
	\begin{itemize}
	 
	  \item M4 (Média de pontos alocados por \textit{sprint}): 15
	  
	  \item M5 (Pontuação das histórias): Tabela \ref{pontuacao_historias}.
	  
	  \item M6 (Repontuação das histórias): Tabela \ref{pontuacao_historias}.
	  
	  \item M7 (Erro na estimativa por história): Tabela \ref{pontuacao_historias}.
	  
	  \item M8 (Precisão da estimativa): Tabela \ref{pontuacao_historias}.
	  
	  \item M9 (Erro médio nas estimativas): 1 SP (aproximado)
	\end{itemize}
	
	\begin{table}
	\centering
	\caption{Pontuação das histórias da Sprint 0.}
	\label{pontuacao_historias}
	\begin{tabular}{ | l | l | l | l | l | }
	\hline
	História & M5 & M6 & M7 & M8 \\ \hline
	US01 & 8 & 5 & -3 & 0.625 \\ \hline
	US02 & 3 & 3 & 0 & 1 \\ \hline
	US03 & 3 & 5 & 2 & 1.666 \\ \hline
	US04 & 3 & 3 & 0 & 1 \\ \hline
	US05 & 3 & 8 & 5 & 2.666 \\ \hline
	US06 & 5 & 5 & 0 & 1 \\ \hline
	US07 & 3 & 3 & 0 & 1 \\ \hline
	US08 & 3 & 3 & 0 & 1 \\ \hline
	US09 & 5 & 5 & 0 & 1 \\ \hline
	US10 & 5 & 5 & 0 & 1 \\ \hline
	US11 & 8 & 5 & -3 & 0.625 \\ \hline
	US12 & 13 & 13 & 0 & 1 \\ \hline
	HT01 & 2 & 2 & 0 & 1 \\ \hline
	HT02 & 2 & 2 & 0 & 1 \\ \hline
	HT03 & 2 & 2 & 0 & 1 \\ \hline
	HT04 & 2 & 2 & 0 & 1 \\ \hline
	\end{tabular}
	\end{table}
	
	\vfill
	\pagebreak
	
	\section{Resposta às questões}
	  
	  Nesta \textit{sprint} foram alocados 15 pontos, referente as histórias US01, US02, HT01 e HT02.
	  
	  \subsection*{Questão 1 - Qual a influência da procrastinação no atraso das entregas?}
	    
	    \subsubsection*{Métricas coletadas}
	    \begin{itemize}
	      \item Valor coletado - \textbf{M1 (\textit{Velocity} diário)}:
	    
		\subitem Sexta - 0
		\subitem Sábado - 0
		\subitem Domingo - 0
		\subitem Segunda - 0
		\subitem Terça - 0
		\subitem Quarta - 2
		\subitem Quinta - 3
		\subitem Sexta - 3
	      
	      \item Valor coletado - \textbf{M2 (Média de pontos realizados por dia)}: 1
	      \item Valor coletado - \textbf{M3 (Pontos planejados por \textit{sprint})}: 15
	    \end{itemize}
	    
	    \subsubsection*{Análise}
	    
	    Realizando a simulação proposta, é possível ver que o escopo proposto para \textit{sprint} iria ser cumprido
	    no último dia da \textit{sprint} caso o trabalho tivesse sido começado no início da \textit{sprint} 
	    (considerando que a equipe realizasse 1 ponto por dia),
	    como mostra a Tabela \ref{sprint0-sim} onde os pontos restantes é negativado na sexta feira.
	    Isto evidencia que a procrastinação afetou consideravelmente nos atrasos para a \textit{sprint 0},
	    pois o escopo proposto seria entregue, caso o trabalho tivesse se iniciado no começo da \textit{sprint}.
		
	    \begin{table}[h]
	    \centering
	    \caption{Dados da simulação para a Sprint 0}
	    \label{sprint0-sim}
	    \begin{tabular}{|l|l|l|}
	    \hline
	    \textbf{Pontos alocados} & \multicolumn{2}{c|}{\textbf{15}}                              \\ \hline
	    \textbf{Dia da sprint}   & \textbf{Pontos consumidos no dia} & \textbf{Pontos restantes} \\ \hline
	    Sexta                    & 1                                 & 14                        \\ \hline
	    Sábado                   & 1                                 & 13                        \\ \hline
	    Domingo                  & 1                                 & 12                        \\ \hline
	    Segunda                  & 1                                 & 11                        \\ \hline
	    Terça                    & 1                                 & 10                        \\ \hline
	    Quarta                   & 3                                 & 7                         \\ \hline
	    Quinta                   & 4                                 & 3                         \\ \hline
	    Sexta                    & 4                                 & -1                        \\ \hline
	    \end{tabular}
	    \end{table}
	  
	  \vfill
	  \pagebreak
	  \subsection*{Questão 2 - Qual a influência dos erros das estimativas no atraso das entregas?}
	    
	    \subsubsection*{Métricas coletadas}
	    
	      \subitem Valor coletado - M3 (Pontos planejados por \textit{sprint}): 15
	      \subitem Valor coletado - M10 (\textit{Velocity} da equipe por \textit{sprint}): 8
	    
	    \subsubsection*{Análise}
	      
	      A Tabela \ref{historias_erro} mostra como ficaria a pontuação das histórias com o erro médio calculado associado a elas.
	    
	      \begin{table}[h]
	      \centering
	      \caption{Pontuação das histórias da Sprint 0 com o erro associado}
	      \label{historias_erro}
	      \begin{tabular}{|l|l|l|l|}
	      \hline
	      \textbf{História} & \textbf{Pontuação das histórias} & \textbf{Pontuação com o erro (+)} & \textbf{Pontuação com o erro (-)} \\ \hline
	      US01              & 8                                & 9                                 & 7                                 \\ \hline
	      US02              & 3                                & 4                                 & 2                                 \\ \hline
	      HT01              & 2                                & 3                                 & 1                                 \\ \hline
	      HT02              & 2                                & 3                                 & 1                                 \\ \hline
	      \multicolumn{2}{|c|}{\textbf{Total}}                 & 19                                & 11                                \\ \hline
	      \end{tabular}
	      \end{table}
	      
	      Mesmo considerando a pontuação mínima com o erro associado, é possível ver que o time não
	      conseguiria entregar o proposto para a \textit{sprint}, uma vez que só conseguiram fazer 
	      8 pontos nesta \textit{sprint}. Isto mostra um possível erro de alocação dos pontos para a \textit{sprint}.
	    
	  \subsection{Questão 3 - Qual a influência da dificuldade na linguagem no atraso das entregas?}
	    
	    Não foi coletada a métrica inerente.
	  
	  \subsection{Questão 4 - Qual a influência da dificuldade de comunicação no atraso das entregas?}
	  
	    Não foi coletada a métrica inerente.
	
\end{apendicesenv}
