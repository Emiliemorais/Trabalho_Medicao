\section{Time de medição (Time GQM)}
    
	O time que realizará as medições, bem como executar o presente plano, é composto pelos seguintes integrantes,
	sendo os dois primeiros integrantes componentes da equipe de gerência do projeto alvo:
	
	\begin{itemize}
	 \item Emilie Morais;
	 \item Ítalo Paiva;
	 \item Leonardo Cambraia;
	 \item Lucas Costa;
	 \item Omar Faria.
	\end{itemize}

	O comprometimento com o processo de medição é facilmente estabelecido, uma vez que dois integrantes do time de GQM 
	fazem parte da gerência do projeto em questão.
      
    \section{Escopo da medição e caracterização da organização}
    
      A organização e o projeto que serão tratados neste plano estão descritos nas seções 
      \ref{organizacao} e \ref{diagnostico}, respectivamente.
    
      \subsection{Área de melhoria}
      
	A organização deseja melhorar as taxas de entregas dos incrementos de \textit{software} nas \textit{sprints}, evitando os
	atrasos, diminuindo assim os custos e riscos para o projeto.
      
    
    \section{Programa de medição}
      
      Para a elaboração do seguinte programa de medição, foi adotado o modelo proposto pelo método GQM (apresentado na seção \ref{gqm}),
      com algumas considerações da Norma ISO/IEC 15939:2007 (apresentada na seção \ref{15939}).
      
      \subsection{Objetivos de medição}
	
	Para o presente trabalho foi definido o objetivo de medição descrito na Tabela \ref{objetivo_medicao}.
	
	\begin{table}[!htb]
	  \centering
	  \caption{Objetivo de medição}
	  \label{objetivo_medicao}
	  \begin{tabular}{|l|l|}
	  \hline
	  \textbf{Analisar}                                                           & as entregas de incremento de software por Sprint \\ \hline
	  \textbf{\begin{tabular}[c]{@{}l@{}}Com o \\ propósito de\end{tabular}}      & melhorar                                         \\ \hline
	  \textbf{Com respeito a}                                                     & atrasos                                          \\ \hline
	  \textbf{\begin{tabular}[c]{@{}l@{}}Sob o ponto \\ de vista de\end{tabular}} & gerentes de projeto                           \\ \hline
	  \textbf{No contexto de}                                                     & projeto proposto de GPP/MDS \\ \hline    	      
	  \end{tabular}
	\end{table}
	
      \subsection{Questões}
	
	Refinando o objetivo de medição definido, obteve-se as seguintes questões:
	
	\begin{itemize}
	 
	 \item \textbf{Q1}. \textit{Qual a influência da procrastinação no atraso das entregas?}
		\subitem A procrastinação gera postergação do trabalho que resulta em sobrecarga ao final da sprint, podendo provocar atraso da entrega das atividades.
	  	 
	 \item \textbf{Q2}. \textit{Qual a influência dos erros das estimativas no atraso das entregas?}
		\subitem Os erros das estimativas provocam prejuízo no entendimento do valor de cada história, prejudicando a divisão das histórias, o valor do trabalho, além de gerar falso entendimento sobre o estado atual do projeto, aumentando a chance de provocar atraso na entrega.
	 
	 \item \textbf{Q3}. \textit{Qual a influência da dificuldade na linguagem no atraso das entregas?}
		\subitem A falta de domínio da linguagem diminui a noção de trabalho que pode ser realizado em determinado tempo, por não se compreender completamente o contexto em que se está trabalhando, aumentando a chance de encontrar novos obstáculos, tendo que despender mais tempo do que o planejado para resolve-los, podendo provocar atrasos na entrega.
	 
	 \item \textbf{Q4}. \textit{Qual a influência da dificuldade de comunicação no atraso das entregas?}
		\subitem Através da comunicação, o time pode se planejar corretamente e se adequar aos problemas reais da sprint. A dificuldade de comunicação proporciona falta de conhecimento sobre o estado estado atual da sprint, caso a percepção de trabalho a realizar seja maior que a esperada pelo time, aumenta a chance de atraso na entrega.
	 
	\end{itemize}

      
      \subsection{Métricas}
	
	Refinando as questões definidas na subseção anterior, obteve-se as seguintes métricas:
	      
	\begin{itemize}
	 
	 \item \textbf{M1}. \textit{Velocity} diário
	 
	   \subitem \textbf{Descrição}: Esta métrica consiste na quantidade pontos que é realizado por dia pelo time.
	   
	   \subitem \textbf{Unidade}: \textit{Story Points}/dia.
	   
	   \subitem \textbf{Forma de coleta}: Acompanhamento diário da produção por planilha online.
	   
	   \subitem \textbf{Questões respondidas}: Esta métrica ajuda a responder à Questão Q1, pois permite acompanhar o quanto
		    foi produzido por dia, proporcionando a avaliação de quando o time começou a trabalhar. Esta métrica pode ser 
		    graficamente acompanhada a partir do gráfico de \textit{burndown}. Esta métrica também é usada para calcular a
		    métrica M2, para realizar a análise proposta.
	 
	 \item \textbf{M2}. Média de pontos realizados por dia
	   
	   \subitem \textbf{Descrição}: Esta métrica é derivada da métrica M1.\textit{Velocity} diário. Consiste na média das velocidades
		    diárias.
	   
	   \subitem \textbf{Unidade}: \textit{Story Points}/dia.
	   
	   \subitem \textbf{Forma de coleta}: Derivada da coleção de métricas M1.
	   
	      \subsubitem \textbf{Fórmula:} 
	      
		$$ M2 = (\sum\limits_{i=1}^{n}M1_i)/n $$
		
	      \subsubitem Onde $n = quantidade\ de\ dias$.
	   
	   \subitem \textbf{Questões respondidas}: Esta métrica ajuda a responder à Questão Q1, pois informa a quantidade média do 
		    que a equipe consegue produzir por dia, servindo para realizar análise proposta para o fator de procrastinação.
	  
	 \item \textbf{M3}. Pontos planejados por \textit{sprint}
	  
	   \subitem \textbf{Descrição}: Esta métrica consiste na quantidade de pontos que foram alocados para cada \textit{sprint}.
	   
	   \subitem \textbf{Unidade}: \textit{Story Points}.
	   
	   \subitem \textbf{Forma de coleta}: Realizando o \textit{planning poker} no planejamento de cada \textit{sprint} 
		    cujos dados são documentados numa planilha online.
	   
	   \subitem \textbf{Questões respondidas}: Esta métrica ajuda a responder às Questões Q1 e Q2, pois fornece os dados para
		    calcular a média dos pontos planejados por \textit{sprint} (M4) que será utilizada para a análise do impacto 
		    da procrastinação e dos erros nas estimativas.
	  
	 \item \textbf{M4}. Média de pontos alocados por \textit{sprint}
	 
	   \subitem \textbf{Descrição}: Esta métrica é derivada da métrica M3.Pontos planejados por \textit{sprint}.
		    Consiste na média dos pontos que são planejados por \textit{sprint}, indicando a quantidade média 
		    de pontos que são alocados.
	   
	   \subitem \textbf{Unidade}: \textit{Story Points}.
	   
	   \subitem \textbf{Forma de coleta}: Derivada da coleção de métricas M3.
	   
	      \subsubitem \textbf{Fórmula:} 
	      
		$$ M4 = (\sum\limits_{i=1}^{n}M3_i)/n $$
		
	      \subsubitem Onde $n = quantidade\ de\ sprints$.
	   
	   \subitem \textbf{Questões respondidas}: Esta métrica ajuda a responder às Questões Q1 e Q2, pois permite a análise 
		    proposta para os fatores de procrastinação e de erros nas estimativas.
	 
	 \item \textbf{M5}. Pontuação das histórias
	  
	   \subitem \textbf{Descrição}: Esta métrica consiste na pontuação inicial de cada história definida no \textit{backlog} no início
		    da \textit{sprint} (na reunião de planejamento).
	   
	   \subitem \textbf{Unidade}: \textit{Story Points}.
	   
	   \subitem \textbf{Forma de coleta}: Realizando o \textit{planning poker} no planejamento da \textit{sprint}
		    cuja história foi alocada. Os dados são documentados numa planilha online.
	   
	   \subitem \textbf{Questões respondidas}: Esta métrica ajuda a responder à Questão Q2, pois fornece parte do insumo para calcular 
		    as métricas M7 e M8. 
		    
	 \item \textbf{M6}. Repontuação das histórias
	   
	   \subitem \textbf{Descrição}: Esta métrica consiste na pontuação final de cada história definida no \textit{backlog} no fim
		    da \textit{sprint} (na reunião de retrospectiva).
	   
	   \subitem \textbf{Unidade}: \textit{Story Points}.
	   
	   \subitem \textbf{Forma de coleta}: Realizando a repontuação utilizando o \textit{planning poker} na retrospectiva da \textit{sprint}
		    cuja história foi alocada. Os dados são documentados numa planilha online.
	   
	   \subitem \textbf{Questões respondidas}: Esta métrica ajuda a responder à Questão Q2, pois fornece a outra parte do insumo para calcular 
		    as métricas M7 e M8.
		    
	 \item \textbf{M7}. Erro na estimativa por história
	 
	   \subitem \textbf{Descrição}: Esta métrica é derivada das métricas M5.Pontuação das histórias e M6.Repontuação das histórias.
		    Consiste na diferença entre os pontos obtidos com a repontuação e os pontos da pontuação inicial.
	   
	   \subitem \textbf{Unidade}: \textit{Story Points}.
	   
	   \subitem \textbf{Forma de coleta}: Derivada das métricas M5 e M6.
	   
	      \subsubitem \textbf{Fórmula:} 
	      
		$$ M7_i = M6_i - M5_i $$
		
	      \subsubitem Onde $i = número da história referente às métricas$
	   
	   \subitem \textbf{Questões respondidas}: Esta métrica ajuda a responder à Questão Q2, pois permite parte da análise 
		    dos erros nas estimativas.
	 
	 \item \textbf{M8}. Precisão da estimativa
	 
	   \subitem \textbf{Descrição}: Esta métrica é derivada das métricas M5.Pontuação das histórias e M6.Repontuação das histórias.
		    Consiste na razão entre os pontos obtidos com a repontuação e os pontos da pontuação inicial. É uma métrica adicional
		    para acrescentar outro ponto de vista para o erro nas estimativas.
	   
	   \subitem \textbf{Unidade}: Adimensional.
	   
	   \subitem \textbf{Forma de coleta}: Derivada das métricas M5 e M6.
	   
	      \subsubitem \textbf{Fórmula:} 
	      
		$$ M8_i = M6_i / M5_i $$
		
	      \subsubitem Onde $i = número\ da\ história\ referente\ às\ métricas$.
	   
	   \subitem \textbf{Questões respondidas}: Esta métrica está atrelada à Questão Q2, pois permite um outro ponto de vista
		    para análise dos erros nas estimativas.
	 
	 \item \textbf{M9}. Erro médio nas estimativas
	   
	   \subitem \textbf{Descrição}: Esta métrica é derivada da métrica M7.Erro na estimativa por história.
		    Consiste na média dos erros obtidos.
	   
	   \subitem \textbf{Unidade}: \textit{Story Points}.
	   
	   \subitem \textbf{Forma de coleta}: Derivada das métricas M7.
	   
	      \subsubitem \textbf{Fórmula:} 
	      
		$$ M9 = (\sum\limits_{i=1}^{n}M7_i)/n $$
		
	      \subsubitem Onde $n = quantidade\ de\ histórias$.
	   
	   \subitem \textbf{Questões respondidas}: Esta métrica ajuda a responder à Questão Q2, pois permite a análise 
		    dos erros nas estimativas, fornecendo um valor médio para os erros.
		    
	 \item \textbf{M10}. \textit{Velocity} da equipe por \textit{sprint}
	 
	   \subitem Explicar a métrica, informando a unidade de medida o método de coleta dessa métrica. Dizer quais questões essa
		    métrica ajuda a responder.
	 
	 \item \textbf{M11}. Percepção do time sobre a linguagem
	 
	   \subitem Explicar a métrica, informando a unidade de medida o método de coleta dessa métrica. Dizer quais questões essa
		    métrica ajuda a responder.
	
	 \item \textbf{M12}. Percepção do time sobre a comunicação
	 
	    \subitem \textbf{Descrição}: Esta métrica representa o grau de percepção do time sobre a comunicação. O conhecimento sobre o estado atual da sprint, os problemas que as duplas estão enfrentando e qual o progresso realizado. O estado atual da comunicação se pode dar em três estados: ruim, boa ou ótima. Para o caso ruim, o time não tem noção do que esta acontecendo dentro de cada par de desenvolvimento, qual o progresso realizado, ou quais problemas estão enfrentando. Para o caso bom, o time se comunica de forma simples sem transparecer o real problema, as duplas tem noção do progresso geral do time através do burndown. No estado ótimo, as duplas se comunicam e compreendem o progresso e problemas de cada uma.
	   
	   \subitem \textbf{Unidade}: Adimensional. 
	   
	   \subitem \textbf{Forma de coleta}: Na reuniões diárias e de retrospectivas.
	   
	   \subitem \textbf{Questões respondidas}: Esta métrica está atrelada à Questão Q4, ela tenta expressar em qual estado a comunicação do time se encontra. Para então propor juntamente com o time de desenvolvimento formas de melhorar a comunicação.
		    
	\end{itemize}
	
      \vfill
      \pagebreak
      \subsection{Modelo GQM proposto}
      
	A Figura \ref{gqm_proposto} ilustra o modelo do programa de medição proposto, contendo o objetivo de medição, as 
	questões associadas e suas respectivas métricas.
	
	\begin{figure}[!htb]
	  \centering
	  \includegraphics[scale=0.27, angle=90]{figuras/GQM}
	  \caption[Modelo do GQM proposto.]{Modelo do GQM proposto.}
	  \label{gqm_proposto}
	\end{figure}
      
    \section{Procedimentos}
      
      Esta seção descreve os procedimentos que devem ser adotados para coleta e análise de dados, e comunicação dos resultados obtidos.
      
      \subsection{Coleta de dados}
      
      \subsection{Análise dos dados}
      
      \subsection{Comunicação dos resultados}