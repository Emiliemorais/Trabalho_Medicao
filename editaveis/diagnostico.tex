Para a realização da pesquisa-ação será considerado um projeto em andamento da organização que possui as seguintes características:


\begin{itemize}
	\item Aplicativo \textit{mobile};
	\item Equipe composta de dois times:
		\subitem 1 time de desenvolvimento com 6 pessoas;
		\subitem 1 equipe de gerência com 6 pessoas;
	\item Membros inexperientes na tecnologia.
	\item Uso de metodologia ágil;
	\item Quantidade de \textit{Sprints}: 7;
	\item Tamanho da \textit{Sprint}: 1 semana;
	\item Dedicação parcial do time, visto que trata-se de uma disciplina.
\end{itemize}



O diagnóstico realizado para ver a situação atual do projeto, selecionado para o estudo, consistiu na análise 
	dos gráficos de \textit{Burndown} das quatro primeiras \textit{Sprints} e na aplicação de um questionário para o 
	time de desenvolvimento.
	
	\subsection{Análise dos gráficos de \textit{Burndown}}

	Para as quatro primeiras \textit{Sprints} é possível ver nas Figuras \ref{fig:burndown0}, \ref{fig:burndown1}, \ref{fig:burndown2}
	\ref{fig:burndown3} que os pontos realizados não correspondem a pontuação total planejada para a \textit{Sprint}. 
	Assim é possível ver que que nem todas as histórias alocadas foram entregues, o que significa que houve atrasos nas entregas.

	O acompanhamento do \textit{Burndown} por \textit{Sprint} auxiliou na visualização da distribuição do trabalho dentro da \textit{Sprint}.
	Assim foi possível ver, por exemplo, que nas \textit{Sprints} 0 e 1 o time demorou para começar a trabalhar, o que pode ter sido uma causa do atraso. Dessa forma, impactando nas \textit{Sprints} seguintes.

	\begin{figure}[!h]
	\begin{minipage}[b]{0.5\linewidth} 
	\centering
	\includegraphics[scale=0.5]{figuras/burndown_sprint0.png}
	\caption{Burndown da \textit{Sprint} 0.}
	\label{fig:burndown0}
	\end{minipage}
	\hspace{0.5cm} 
	\begin{minipage}[b]{0.5\linewidth}
	\centering
	\includegraphics[scale=0.5]{figuras/burndown_sprint1.png}
	\caption{Burndown da \textit{Sprint} 1.}
	\label{fig:burndown1}
	\end{minipage}
	\end{figure}
	
	\begin{figure}[!h]
	\begin{minipage}[b]{0.5\linewidth} 
	\centering
	\includegraphics[scale=0.5]{figuras/burndown_sprint2.png}
	\caption{Burndown da \textit{Sprint} 2.}
	\label{fig:burndown2}
	\end{minipage}
	\hspace{0.5cm} 
	\begin{minipage}[b]{0.5\linewidth}
	\centering
	\includegraphics[scale=0.5]{figuras/burndown_sprint3.png}
	\caption{Burndown da \textit{Sprint} 3.}
	\label{fig:burndown3}
	\end{minipage}
	\end{figure}

	\vfill
	\pagebreak

	\subsection{Questionário}

	A fim de coletar dados sobre a percepção dos membros sobre possíveis aspectos que afetaram a entrega nos prazos foi aplicado
	um questionário ao time de desenvolvimento que se encontra no apêndice \ref{apendice:questionario}.

	No apêndice pode ser visto que o questionário é composto de afirmações que são avaliadas pelo respondente em uma escala de 1 a 5. Na qual 1 significa discordância total da afirmação e 5 concordância total. 

	As respostas coletadas foram contabilizadas de acordo com cada categoria da escala para cada afirmação. Cada afirmação tratava de um fator que poderia estar ocasionando os atrasos. Essas afirmações são:

	\begin{table}[h!]
	\centering
	\caption{Afirmações do questionário e fatores associados}
	\label{my-label}
	\begin{tabular}{|l|l|}
	\hline
	\multicolumn{1}{|c|}{\textbf{Afirmação}}                                                     & \multicolumn{1}{c|}{\textbf{Fator}}                                                       \\ \hline
	\begin{tabular}[c]{@{}l@{}}1. Os atrasos ocorridos decorreram dos\\  erros nas estimativas em StoryPoints.\end{tabular}                 & Erro nas estimativas                                                                      \\ \hline
	\begin{tabular}[c]{@{}l@{}}2. A dificuldade na linguagem ocasionou\\  os atrasos nas entregas da \textit{Sprint}.\end{tabular}        & Dificuldade na linguagem                                                                  \\ \hline
	\begin{tabular}[c]{@{}l@{}}3. O tamanho da \textit{Sprint}\\  afetou nas entregas no prazo.\end{tabular}                              & Tamanho da \textit{Sprint}                                                              \\ \hline
	\begin{tabular}[c]{@{}l@{}}4. Os atrasos decorreram da mudança\\  de metodologia, devido ao período de adaptação.\end{tabular}          & \begin{tabular}[c]{@{}l@{}}Dificuldade na adaptação \\ para metodologia ágil\end{tabular} \\ \hline
	\begin{tabular}[c]{@{}l@{}}5. A demora na iniciação do desenvolvimento na\\  \textit{Sprint} ocasionou os atrasos.\end{tabular}       & Procrastinação                                                                            \\ \hline
	\begin{tabular}[c]{@{}l@{}}6. A dificuldade na comunicação afetou\\  a entrega no prazo.\end{tabular}                                   & Dificuldade na comunicação                                                                \\ \hline
	\begin{tabular}[c]{@{}l@{}}7. A dedicação parcial do time afetou\\  a entrega no prazo.\end{tabular}                                    & Dedicação parcial do time                                                                 \\ \hline
	\begin{tabular}[c]{@{}l@{}}8. A dificuldade no entendimento das histórias\\  acarretou nos atrasos nas \textit{sprints}.\end{tabular} & \begin{tabular}[c]{@{}l@{}}Dificuldade no entendimento\\  das histórias\end{tabular}      \\ \hline
	\end{tabular}
	\end{table}

	A sumarização dos resultados pode ser vista na Tabela \ref{tab:analise_respostas}.	Para uma melhor visualização das respostas também foram gerados gráficos, que também podem ser vistos em apêndice (Apêndice \ref{apendice:graficos}).

	\begin{table}[!h]
	\flushleft
	\caption{Percentual de respostas por categoria em cada pergunta}
	\label{tab:analise_respostas}
	\begin{tabular}{|c|l|l|l|l|l|l|l|l|}
	\hline
	\textbf{\begin{tabular}[c]{@{}c@{}}Afirmação/\\ Categorias\end{tabular}} & \textbf{1} & \textbf{2} & \textbf{3} & \textbf{4} & \textbf{5} & \textbf{6} & \textbf{7} & \textbf{8} \\ \hline
	\begin{tabular}[c]{@{}c@{}}Discordo\\  totalmente\end{tabular}   & 0\%        & 0\%        & 16,67\%    & 33,33\%    & 0\%        & 0\%        & 16,67\%    & 100\%      \\ \hline
	\begin{tabular}[c]{@{}c@{}}Discordo\\  parcialmente\end{tabular} & 0\%        & 0\%        & 33,33\%    & 33,33\%    & 0\%        & 16,67\%    & 50\%       & 0\%        \\ \hline
	Neutro                                                           & 33,33\%    & 0\%        & 33,33\%    & 33,33\%    & 16,67\%    & 33,3\%     & 16,67\%    & 0\%        \\ \hline
	\begin{tabular}[c]{@{}c@{}}Concordo\\  parcialmente\end{tabular} & 50\%       & 50\%       & 16,67\%    & 0\%        & 66,67\%    & 50\%       & 0\%        & 0\%        \\ \hline
	\begin{tabular}[c]{@{}c@{}}Concordo\\  totalmente\end{tabular}   & 16,67\%    & 50\%       & 0\%        & 0\%        & 16,67\%    & 0\%        & 16,67\%    & 0\%        \\ \hline
	\end{tabular}
	\end{table}



		\subsubsection{Análise das respostas}

			Analisando os dados apresentados na tabela \ref{tab:analise_respostas} e através do acompanhamento das \textit{sprints}, é possível analisar os fatores que a equipe acredita que tenham ocasionado os atrasos.
	
			Dessa forma, é possível ver que a equipe considera que errou nas estimativas em \textit{StoryPoints} e isso afetou nas entregas das histórias, visto que a maioria respostas convergiu para a concordância com a afirmação. Esse fator é um fator provável dentro da organização considerada nesse trabalho, visto que as equipes são inexperientes.

			Outro fator que convergiu para a concordância foi a dificuldade na linguagem. O time analisa como um grande fator que impediu o desenvolvimento das histórias no prazo.

			O tamanho da \textit{sprint} foi um fator que não houve uma convergência nítida, todavia a maioria discordou ou se colocou como neutro em relação a esse fator. Assim, não é um fator que tem um peso sobre os atrasos ocorridos.

			A adaptação para a nova metodologia não foi um fator responsável pelos atrasos visto que as respostas convergiram para a discordância da afirmação.

			A demora na iniciação do desenvolvimento, ou seja, a procrastinação da equipe foi considerado um fator responsável pelos atrasos, visto que as respostas convergiram para a concordância.

			Em relação à dificuldade na comunicação a equipe se posicionou, em sua maioria, neutra ou em uma concordância parcial. A partir disso é possível
			ver que a dificuldade de comunicação afeta os atrasos mas com um peso bem menor que outros fatores.

			Para o fator dedicação parcial é possível que a equipe, em sua maioria, não acredita ser um problema que tenha ocasionado os atrasos.

			Por fim, a dificuldade no entendimento das histórias foi considerado um fator sem influência nos atrasos. Isso é perceptível já que na organização, objeto de pesquisa deste trabalho, os requisitos da aplicação são definidos pelo time.

		\subsubsection{Resultado da análise}
			Resumindo a análise dos resultados, temos que os fatores que foram considerados responsáveis pelos atrasos, que serão objetos de estudo deste trabalho, são:

			\begin{itemize}
				\item Erros nas estimativas;
				\item Dificuldade na linguagem;
				\item Procrastinação;
				\item Dificuldade de comunicação;
			\end{itemize}